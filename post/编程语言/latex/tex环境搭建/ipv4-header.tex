\documentclass[tikz,border=5mm]{standalone}
\usepackage{ctex}
\usepackage{tikz}
\usepackage{svg}
% 设置主字体
%\setmainfont{FangSong}
% \setCJKmainfont{FangSong}
 
\usetikzlibrary{positioning, shapes.multipart}
\begin{document}
	\begin{tikzpicture}[
		box/.style={
			draw, 
			minimum width=32cm,
			minimum height=1.5cm,
			align=center,
			inner sep=0
		},
		field/.style={
			draw, 
			minimum height=1.5cm,
			align=center,
			font=\normalsize,
			inner sep=0
		}
	]
		
		% \node [选项] (名称) at (坐标) {内容}; 默认绘制的为矩形	
		% 绘制32位行框架
		%\foreach \i in {0,1,2,3,4} {
		%	\node[box] (row\i) at (0,-\i*1.5) {};
		%	\node[left] at (row\i.west) {\i};
		%}
		% 确定原点位置
		%\draw[fill=red] (0,-3.75) circle (2pt) node[right] {原点};
		
		% 位数表示
		\node[minimum width=4cm] (left0) at (-15.5,1.0) {0};
		\node[minimum width=4cm] (left0) at (-0.5,1.0) {15};
		\node[minimum width=4cm] (left0) at (0.5,1.0) {16};
		\node[minimum width=4cm] (left0) at (15.5,1.0) {31};
		% 绘制右侧指示箭头和中间数字
		\node (A) at (17.0, 0.75) {};
		\node (B) at (17.0,-2.75) {};
		\draw[->,line width=1pt] (B) -- (A);
		\node at (17.0,-3) {20字节}; % 默认显示
		\node (C) at (17.0, -3.25) {};
		\node (D) at (17.0,-6.75) {};
		\draw[->,line width=1pt] (C) -- (D);
		
		\node (E) at (16.25, 0.75) {};
		\node (F) at (17.75, 0.75) {};
		\draw[-,line width=1.2pt] (E) -- (F);
		
		\node (G) at (16.25, -6.75) {};
		\node (H) at (17.75, -6.75) {};
		\draw[-,line width=1.2pt] (G) -- (H);

		% 第一行字段
		\node[field, minimum width=4cm] (version) at (-14,0) {版本\\4位};
		\node[field, minimum width=4cm, right=0pt of version] (ihl) {首部长度\\4位};
		\node[field, minimum width=8cm, right=0pt of ihl] (tos) {服务类型\\8位};
		\node[field, minimum width=16cm, right=0pt of tos] (totlen) {总长度\\16位};
		
		% 第二行字段
		\node[field, minimum width=16cm] (id) at (-8,-1.5) {标识符\\16位};
		\node[field, minimum width=3cm, right=0pt of id] (flags) {标志\\3位};
		\node[field, minimum width=13cm, right=0pt of flags] (frag) {分片偏移\\13位};
		
		% 第三行字段
		\node[field, minimum width=8cm] (ttl) at (-12,-3.0) {生存时间\\8位};
		\node[field, minimum width=8cm, right=0pt of ttl] (proto) {协议\\8位};
		\node[field, minimum width=16cm, right=0pt of proto] (checksum) {首部校验和\\16位};
		
		% 第四行字段
		\node[field, minimum width=32cm] (src) at (0,-4.5) {源IP地址\\32位};
		
		% 第五行字段
		\node[field, minimum width=32cm] (dst) at (0,-6.0) {目的IP地址\\32位};
		
		% 第六行字段
		\node[field, minimum width=32cm, minimum height=2.5cm] (dst) at (0,-8.0) {选项(如果有)};
		
		%% 第七行字段
		\node[field, minimum width=32cm, minimum height=2.5cm] (dst) at (0,-10.5) {数据};
		
	
	\end{tikzpicture}
	% 导出当前页面为SVG(需shell逃逸)
	\immediate\write18{pdf2svg \jobname.pdf \jobname.svg}
\end{document}
